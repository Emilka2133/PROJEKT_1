\documentclass[11pt,a4paper]{article}

\usepackage{amsmath,amssymb}
\usepackage[T1]{fontenc}
\usepackage[utf8]{inputenc}
\usepackage[polish]{babel}
\usepackage{graphicx,lmodern}
\usepackage{graphics}
%bibliografia
\usepackage[comma,round]{natbib}
\bibliographystyle{abrvnat}

\usepackage{lipsum}
%---METADANE
\title{Sprawozdanie}
\author{Magdalena Frąckiewicz i Emilia Felczak}
\date{\today}
%----START DOKUMENTU
\begin{document}
\maketitle 
\tableofcontents 
\newpage
\section {Cel ćwiczenia}
Celem ćwiczenia jest stworzenie skryptu implementującego transformacje:\\
-zamiana współrzędnych XYZ na flh\\
-zamiana współrzędnych flh na XYZ\\
-zamiana współrzędnych XYZ na neu\\
-zamiana współrzędnych flh na XY w układzie 2000\\
-zamiana współrzędnych flh na XY w układzie 1992\\

\section {Wykorzystane narzędzia i materiały}
Narzędzia i materiały wykorzystane do replikacji ćwiczenia:\\
-Python 3.11  (64-bit)\\                                                       
-Python 3.12 (64-bit)  \\                                                                 
-Spyder  \\                                                                              
-GitHub  \\                                                                             
-Wiersz polecenia\\
-System operacyjny  Microsoft Windows 11\\
- Biblioteka numpy, math\\
- prezentacje z zajęć (w tym kody z poprzedniego semestru)\\


\section {Przebieg ćwiczenia}
Pierwszym krokiem było stworzenie nowego repozytorium w GitHubie, aby móc wykonywać commity. Dzięki temu wspólnie mogłyśmy pracować na jednym pliku z różnych komputerów. Następnie storzyłyśmy klasę o nazwie Transformacje, która umożliwiła korzystanie z kilku elipsoid i odpowiednio przypisanych do nich parametrów za pomocą "self.". Importowałyśmy potrzebne biblioteki takie jak: numpy, math, sys. Biblioteka numpy  potrzebna jest do obliczeń macierzowych aby uzyskać precyzyjniejsze wyniki i schludniejszy kod. Kolejnym krokiem było napisanie funkcji zamieniających współrzędne. Na podstawie pliku wyjściowego (wsp\_inp.txt) mogłyśmy sprawdzać poprawność wyników. Było to możliwe dzięki wprowadzeniu do kodu odczytu danych z pliku .txt oraz utworzenia nowego pliku z wynikami. Po zweryfikowaniu działania kodu zajęłyśmy się stylistyką pliku wyjściowego oraz za pomocą sys.argv dałyśmy możliwość korzystania z kodu w wierszu poleceń z zainstalowaną biblioteką numpy. Użytkownik może wybrać flagę opisującą daną funkcję, elipsoidę odniesienia, liczbę wierszy nagłówka w pliku wejściowym oraz plik wejściowy .txt. W rezultacie tworzy się plik wynikowy w tym samym folderze co kod (kod1.py). 

\section {Podsumowanie}

\subsubsection{Rezultat}
\subsubsection{Nabyte umiejętności}
Umiejętności nabyte w trakcie wykonywania ćwiczenia:\\
 - pisanie kodu obiektowego w Pythonie\\
- implementowanie algorytmów pochodzących ze źródeł zewnętrznych (tj. takich, których nie wymyśliliśmy sami)\\
- tworzenie dokumentów w latex\\
- współpraca w dwuosobowym zespole z wykorzystaniem systemu kontroli wersji git (tworzenie repozytorium w GitHub, wykonywanie commitów)\\
- tworzenie narzędzi w interfejsie tekstowym (cli) potrafiących przyjmować argumenty przy wywołaniu (program jest w stanie pobierać dane z pliku tekstowego)\\
- pisanie dokumentacji (parametry, wyniki, przebieg i opis funkcji\\
- pisanie Readme w GitHub.md\\

\subsubsection{Spostrzeżenia i trudności }
\end{document}